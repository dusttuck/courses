\input logicdocument
\title{Phil 210 - Problem Set 3}

\heading
Problem Set 3
Your Name
Introduction to Formal Logic
\endheading

Construct truth tables for the following arguments. For each one, state whether it's valid or invalid.

\problems
\problem{Example}
\argument
 A ^ -B
 C v A
\argumentline
 -[C v B]
\endargument
	\answer
	\truthtable
	 A | B | C : A. ^ .-.B | C. v .A | -.[.C. v .B.]
	\truthtableline
	 T | T | T : t.*F .F.t | t.*T .t |*F.[.t. T .t.]
	 T | T | F : t.*F .F.t | f.*T .t |*F.[.f. T .t.]
	 T | F | T : t.*T .T.f | t.*T .t |*F.[.t. T .f.]
	 T | F | F : t.*T .T.f | f.*T .t |*T.[.f. F .f.]
	 F | T | T : f.*F .F.t | t.*T .f |*F.[.t. T .t.]
	 F | T | F : f.*F .F.t | f.*F .f |*F.[.f. T .t.]
	 F | F | T : f.*F .T.f | t.*T .f |*F.[.t. T .f.]
	 F | F | F : f.*F .T.f | f.*F .f |*T.[.f. F .f.]
	\endtruthtable

	Invalid. (You don't need to say why, but it's because row 3 is T T F.)
	\endanswer

\problem{1}
\argument
 -A
 -[B ^ A]
\argumentline
 -B
\endargument
	\answer
	\truthtable
	 A | B : -.A | -.[.B. ^ .A.] | -.B
	\truthtableline
	   |   :  .  |  .[. .   . .] |  . 
	\endtruthtable
	Valid/Invalid?
	\endanswer

\problem{2}
\argument
 A v B
 -A v C
 -B v C
\argumentline
 C
\endargument
	\answer
	\truthtable
	 A | B | C : A. v .B | -.A. v .C | -.B. v .C | C
	\truthtableline
	   |   |   :  .   .  |  . .   .  |  . .   .  |  
	\endtruthtable
	Valid/Invalid?
	\endanswer

\problem{3}
\argument
 -*[[A ^ B] v -A*]
 [-B v A] ^ B
\argumentline
 A v -[B v A]
\endargument
	\answer
	\truthtable
	 A | B : -.*[[.A. ^ .B.]. v .-.A.*] | [.-.B. v .A.]. ^ .B | A. v .-.[.B. v .A.]
	\truthtableline
	   |   :  . [[. .   . .].   . . . ] | [. . .   . .].   .  |  .   . .[. .   . .]
	\endtruthtable
	Valid/Invalid?
	\endanswer
\endproblems

For each of the following arguments, use an abbreviated truth table to determine whether it's valid or invalid. State your decision.

\problems
\problem{Example}
\argument
 A v -C
 -[A v B]
\argumentline
 -[B v C]
\endargument
	\answer
	\truthtable
	 A. v .-.C | -.[.A. v .B.] | -.[.B. v .C.]
	\truthtableline
	  . T . .  | T. . .   . .  | F. . .   . .
	  .   . .  |  . . . F . .  |  . . . T . .
	  .   . .  |  . .*F.  .*F. |  . . .   . .
	 F.   . .  |  . . .   . .  |  . .F.   . .
	  .   .T.  |  . . .   . .  |  . . .   .*T.
	  .  ..T/*F|  . . .   . .  |  . . .   . .
	\endtruthtable
	Valid. (You don't need to say why, but it's because when we attempted to make both premises T and the conclusion F at the same time, we wound up contradicting ourselves.)
	\endanswer

\problem{4}
\argument
 A v B
 -[C ^ B]
\argumentline
 -C
\endargument
	\answer
	\truthtable
	 A. v .B | -.[.C. ^ .B.] | -.C
	\truthtableline
	  . T .  | T. . .   . .  | F.
	  .   .  |  . . .   . .  |  .
	\endtruthtable
	Valid/Invalid?
	\endanswer

\problem{5}
\argument
 A ^ [B v C]
 -B ^ -[A ^ D]
\argumentline
 -[D v B] ^ C
\endargument
	\answer
	\truthtable
	 A. ^ .[.B. v .C.] | -.B. ^ .-.[.A. ^ .D.] | -.[.D. v .B.]. ^ .C
	\truthtableline
	  . T . . .   . .  |  . . T . . . .   . .  |  . . .   . . . F . 
	  .   . . .   . .  |  . .   . . . .   . .  |  . . .   . . .   . 
	\endtruthtable
	Valid/Invalid?
	\endanswer

\endproblems

For each of the following arguments,
\list
a: Use an abbreviated truth table to determine whether it's valid or invalid, given only the connectives. State your decision.

b: If it's invalid, attempt to build a height model that's a counterexample. If the argument is valid, or if no countermodel can be constructed, just say so.

c: State whether the argument is valid or invalid, given both the connectives and the predicates.
\endlist

\problems
\problem{Example}
\argument
 Taller(ruth,alonzo) v Taller(ruth,kurt)
 -[Shorter(kurt,alonzo) ^ Taller(ruth,kurt)]
\argumentline
 Taller(ruth,kurt) v Shorter(kurt,alonzo)
\endargument
	\answerlist
	a:
	\truthtable
	 T(r,a). v .T(r,k) | -.[.S(k,a). ^ .T(r,a).] | T(r,k). v .S(k,a)
	\truthtableline
	       . T .       | T. .      .   .      .  |       . F .
	       .   .       |  . .      . F .      .  |   *F  .   .  *F
	       .   .   F   |  . . F    .   .      .  |       .   .
	   *T  .   .       |  . .      .   .      .  |       .   .
	       .   .       |  . .      .   .   T  .  |       .   .
	\endtruthtable
	Invalid, given only the connectives.

	b:
	\heightmodel
	 Alonzo < Ruth = Kurt
	\endheightmodel
	\OR
	\heightmodel
	 Alonzo < Ruth < Kurt
	\endheightmodel

:	(You only need to give one countermodel, but I've given two for illustration.)

	c: Invalid, given the connectives and predicates.

:	(You don't need to say why, but it's because the height model depicts a situation in which $T(r,a)$ is T, $T(r,k)$ is F, and $S(k,a)$ is F, and we know from the abbreviated truth table that this makes both premises T and the conclusion F.)

	\endanswerlist

\problem{6}
\argument
 Shorter(ruth,kurt) v Taller(ruth,alonzo)
 -[Taller(alonzo,kurt) ^ Taller(ruth,alonzo)]
\argumentline
 -Taller(alonzo,kurt)
\endargument
	\answerlist
	a:
	\truthtable
	 S(r,k). v .T(r,a) | -.[.T(a,k). ^ .T(r,a).] | -.T(a,k)
	\truthtableline
	       .   .       |  . .      .   .      .  |  .
	\endtruthtable
	Valid/Invalid?, given only the connectives.

	b: It's valid.\OR No countermodel is possible.\OR
	\heightmodel
	 
	\endheightmodel

	c: Valid/Invalid?, given the connectives and predicates.
	\endanswerlist

\problem{7}
\argument
 Taller(ruth,alonzo) ^ Taller(alonzo,kurt)
 -Taller(ruth,alonzo) v Taller(alonzo,kurt)
\argumentline
 Taller(ruth,kurt) v -Taller(alonzo,kurt) 
\endargument
	\answerlist
	a:
	\truthtable
	 T(r,a). ^ .T(a,k) | -.T(r,a). v .T(a,k) | T(r,k). v .-.T(a,k)
	\truthtableline
	       .   .       |  .      .   .       |       .   . .
	\endtruthtable
	Valid/Invalid?, given the connectives.

	b: It's valid.\OR No countermodel is possible.\OR
	\heightmodel
	 
	\endheightmodel

	c: Valid/Invalid?, given the connectives and predicates.
	\endanswerlist

\endproblems
\bye
