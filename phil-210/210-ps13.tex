\input logicdocument
\title{Phil 210 - Week 13 Problem Set}

\heading
Week 13 Problem Set
Your Name
Introduction to Formal Logic
\endheading

For each of the following valid arguments, complete a Fitch-style natural deduction to prove its validity.

\quantifiers
\problems
\problem{1}
\argument
 Ax[Logician(x) > Happy(x)]
 Logician(ruth) 
\argumentline
 Happy(ruth) 
\endargument
	\answer
	\fitchproof
	 1. | Ax[L(x) > H(x)]
	 2. | L(r)             ..  Goal: H(r)
	    |---
	 3. | L( ) > H( )      .  A Elim: 1
	\endfitchproof
	\endanswer

\problem{2}
\argument
 Ax[Logician(x) > Happy(x)]
 -Happy(ruth) 
\argumentline
 -Logician(ruth) 
\endargument
\Hint Begin by setting up $-$ Intro.
	\answer
	\fitchproof
	 1. | Ax[L(x) > H(x)]
	 2. | -H(r)             ..  Goal: -L(r)
	    |---
	 3. | 
	\endfitchproof
	\endanswer

\problem{3}
\argument
 Ax[Logician(x) > Happy(x)]
 Logician(ruth)
 Ex Happy(x) > Happy(alonzo)
\argumentline
 Happy(alonzo) 
\endargument
\Hint Premise 3 is a $>$ sentence, so don't use $E$ Elim.
	\answer
	\fitchproof
	 1. | Ax[L(x) > H(x)]
	 2. | L(r)
	 3. | ExH(x) > H(a)    ..  Goal: H(a)
	    |---
	 4. |
	\endfitchproof
	\endanswer

\problem{4}
\argument
 Ax[Logician(x) > Happy(x)]
 Logician(ruth) v Logician(alonzo)
\argumentline
 Ex Happy(x)
\endargument
\Hint You can do $A$ Elim multiple times.
	\answer
	\fitchproof
	 1. | Ax[L(x) > H(x)]
	 2. | L(r) v L(a)      ..  Goal: ExH(x)
	    |---
	 3. | 
	\endfitchproof
	\endanswer

\problem{5}
\argument
 Ax[Logician(x) > Happy(x)]
 Ex Logician(x)
\argumentline
 Ex Happy(x)
\endargument
\Hint This is the first time you have to use $E$ Elim.
	\answer
	\fitchproof
	 1. | Ax[L(x) > H(x)]
	 2. | ExL(x)           ..  Goal: ExH(x)
	    |---
	 3. |   | :a: L(a)     ..  Assumption  .  Setting up: E Elim  .  Goal: ExH(x)
	    |   |---
	 4. |   |
	\endfitchproof
	\endanswer

\problem{6}
\argument
 Ex-Happy(x)
 Ax[Happy(x) v Sad(x)]
\argumentline
 Ex Sad(x)
\endargument
	\answer
	\fitchproof
	 1. | Ex-H(x)
	 2. | Ax[H(x) v S(x)]  ..  Goal: ExS(x)
	    |---
	 3. | 
	\endfitchproof
	\endanswer

\endproblems
\bye
