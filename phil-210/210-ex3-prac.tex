\input logicdocument
\title{Phil 210 - Practice Final Exam}

\heading
Practice Final Exam
Your Name
Introduction to Formal Logic
\endheading

(40 points.) Translate the following English sentences into the language of first-order logic. (To shake things up a little bit, these sentences are about me, my wife, and my sisters-in-law instead of the usual collection of logicians.)

\quantifiers
\problems
\problem{1} (9 points.)
Dustin loves and is married to Gia.
	\answer
	$ $
	\endanswer

\problem{2} (8 points.)
Nicole is married to neither Dustin nor Gia.
	\answer
	$ $
	\endanswer

\problem{3} (7 points.)
Angela is married to Nicole, and she's happy if she sees her.
	\answer
	$ $
	\endanswer

\problem{4} (5 points.)
Every dog is happy.
	\answer
	$ $
	\endanswer

\problem{5} (4 points.)
No happy dogs are barking.
	\answer
	$ $
	\endanswer

\problem{6} (3 points.)
Gia loves two cats.
	\answer
	$ $
	\endanswer

\problem{7} (3 points.)
Every dog saw a cat.
	\answer
	$ $
	\endanswer

\endproblems

(10 points.) Use a full or abbreviated truth table (your choice) to determine whether the following argument is valid or invalid and state your answer.

\problems
\problem{8}
\argument
 *[[B ^ C] > D*]
 -[D v C]
\argumentline
 -B
\endargument
	\answer
	\truthtable
	 B | C | D : *[[.B. ^ .C.]. > .D.*] | -.[.D. v .C.] | -.B
	\truthtableline
	 T | T | T :  [[.t. T .t.].*T .t. ] |*F.[.t. T .t.] |*F.t
	 T | T | F :  [[.t. T .t.].*F .f. ] |*F.[.f. T .t.] |*F.t
	 T | F | T :  [[.t. F .f.].*T .t. ] |*F.[.t. T .f.] |*F.t
	 T | F | F :  [[.t. F .f.].*T .f. ] |*T.[.f. F .f.] |*F.t
	 F | T | T :  [[.f. F .t.].*T .t. ] |*F.[.t. T .t.] |*T.f
	 F | T | F :  [[.f. F .t.].*T .f. ] |*F.[.f. T .t.] |*T.f
	 F | F | T :  [[.f. F .f.].*T .t. ] |*F.[.t. T .f.] |*T.f
	 F | F | F :  [[.f. F .f.].*T .f. ] |*T.[.f. F .f.] |*T.f
	\endtruthtable
	\truthtable
	 B | C | D : *[[.B. ^ .C.]. > .D.*] | -.[.D. v .C.] | -.B
	\truthtableline
	   |   |   :  [[. .   . .]. T . . ] | T.[. .   . .] | F. 
	   |   |   :  [[. .   . .].   . . ] |  .[. . F . .] |  .*T
	   |   |   :  [[.T.   . .].   . . ] |  .[.*F. .*F.] |  . 
	   |   |   :  [[. .   .F.].   .F. ] |  .[. .   . .] |  . 
	   |   |   :  [[. . F . .].   . . ] |  .[. .   . .] |  . 
	\endtruthtable
	Invalid (in the full table, it's because row 4 is T T F).
	\endanswer

\endproblems

(6 points.) For each of the following sentences, state whether it's true or false in this model:
\answer
	\firstordermodel 
	Domain: Ruth, Alonzo, Kurt, Irene

	Dog:    Alonzo, Irene
	Happy:  Ruth, Alonzo, Kurt
	Saw:    <Ruth, Ruth>, <Ruth, Alonzo>, <Ruth, Kurt>,
	   .    <Alonzo, Alonzo>, <Alonzo, Irene>,
	   .    <Kurt, Ruth>, <Kurt, Kurt>,
	   .    <Irene, Ruth>, <Irene, Alonzo>, <Irene, Kurt>
	\endfirstordermodel
\endanswer\bigskip

\problems
\problem{9}
$ Ex[Dog(x) ^ Happy(x)] $
	\answer
	 T / F
	\endanswer

\problem{10}
$ Ax[Dog(x) > Happy(x)] $
	\answer
	 T / F
	\endanswer

\problem{11}
$ Ax*[Happy(x) > Ey[Dog(y) ^ Saw(x,y)]*] $
	\answer
	 T / F
	\endanswer
\endproblems

(4 points.) Give a countermodel for the following invalid argument.

\problems
\problem{12}
\argument
 Ax[Person(x) > Smiling(x)]
 Ex[Smiling(x) ^ Working(x)]
\argumentline
 Ax[Person(x) > Working(x)]
\endargument
	\answer
	\firstordermodel
	Domain:  

	Person:  
	Smiling: 
	Working: 
	\endfirstordermodel
	\endanswer
\endproblems 

(40 points.) For each of the following valid arguments, construct a Fitch-style natural deduction to prove its validity.

\problems
\problem{13}
\argument
 B > C
 C > D
 B ^ F
\argumentline
 G v D
\endargument
\points{12}
	\answer
	\fitchproof
	 1. | B > C
	 2. | C > D
	 3. | B ^ F  ..  Goal: G v D
	    |---
	 4. | 
	\endfitchproof
	\endanswer

\problem{14}
\argument
 P(a) > Q(a)
 Q(b) ^ P(c)
\argumentline
 a=c > Q(c)
\endargument
\points{10}
	\answer
	\fitchproof
	 1. | P(a) > Q(a)
	 2. | Q(b) ^ P(c)  ..  Goal: a=c > Q(c)
	    |---
	 3. | 
	\endfitchproof
	\endanswer

\problem{15}
\argument
 [B ^ C] > D
 B > [C ^ -D]
\argumentline
 -B
\endargument
\points{8}
	\answer
	\fitchproof
	 1. | [B ^ C] > D
	 2. | B > [C ^ -D]  ..  Goal: -B
	    |---
	 3. | 
	\endfitchproof
	\endanswer

\problem{16}
\argument
 Ax[Saw(ruth,x) > H(x)]
 Ex Saw(ruth,x)
\argumentline
 Ex[Saw(ruth,x) ^ H(x)]
\endargument
\points{6}
	\answer
	\fitchproof
	 1. | Ax[S(r,x) > H(x)]
	 2. | Ex S(r,x)              ..  Goal: Ex[S(r,x) ^ H(x)]
	    |---
	 3. | 
	\endfitchproof
	\endanswer

\problem{17}
\argument
 Ax[Saw(x,ruth) v Saw(ruth,x)]
 AxAy[Saw(x,y) > Saw(y,x)]
\argumentline
 Ax Saw(x,ruth)
\endargument
\points{4}
	\answer
	\fitchproof
	 1. | Ax[S(x,r) v S(r,x)]
	 2. | AxAy[S(x,y) > S(y,x)]        ..  Goal: Ax S(x,r)
	    |---
	 3. | 
	\endfitchproof
	\endanswer

\endproblems
\bye
