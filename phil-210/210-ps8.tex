\input logicdocument
\title{Phil 210 - Problem Set 8}

\heading
Problem Set 8
Your Name
Introduction to Formal Logic
\endheading

For each of the following valid arguments, complete a Fitch-style natural deduction to prove its validity. Do this by filling in the missing lines, citations, or both.

\problems
\problem{1}
\argument
 A
 -[A ^ B]
\argumentline
 -B
\endargument
	\answer
	\fitchproof
	 1. | A
	 2. | -[A ^ B]   ..  Goal: -B
	    |---
	 3. |   | B      ..  Assumption  .  Setting up: - Intro  .  Goal: !
	    |   |---
	 4. |   |        .  ^ Intro: 1, 3
	 5. |   | !      .  ! Intro: 2, 4
	    |   +
	 6. |            .  - Intro: 3-5
	\endfitchproof
	\endanswer

\problem{2}
\argument
 A > B
 B > C
 -C
\argumentline
 -A
\endargument
	\answer
	\fitchproof
	 1. | A > B
	 2. | B > C
	 3. | -C      ..  Goal: -A
	    |---
	 4. |   | A   ..  Assumption  .  Setting up: - Intro  .  Goal: !
	    |   |---
	 5. |   | 
	\endfitchproof
	\endanswer

\problem{3}
\argument
 -A > -B
 B
\argumentline
 A
\endargument
\Hint You have to prove $--A$ first (the ``last resort'' option).
	\answer
	\fitchproof
	 1. | -A > -B
	 2. | B        ..  Goal: A
	    |---
	 3. |   | -A   ..  Assumption  .  Setting up: - Intro  .  Goal: !
	    |   |---
	 4. |   |      .  
	 5. |   |      .  
	    |   +
	 6. | --A      .  - Intro: 3-5
	 7. | A        .  - Elim: 6
	\endfitchproof
	\endanswer

\problem{4}
\argument
 [A ^ B] > C
 -C
\argumentline
 A > -B
\endargument
	\answer
	\fitchproof
	 1. | [A ^ B] > C
	 2. | -C            .. Goal: A > -B
	    |---
	 3. |   | A         ..  Assumption  .  Setting up: > Intro  .  Goal: -B
	    |   |---
	 4. |   | 
	\endfitchproof
	\endanswer

\widerfitchsetup
\problem{5}
\argument
 A v B
 -B
\argumentline
 A
\endargument
\Hint Problems 5--7 are the first times you have to use $!$ Elim.
	\answer
	\fitchproof
	 1. | A v B
	 2. | -B      ..  Goal: A
	    |---
	 3. |   | A   ..  Assumption  .  Setting up: v Elim: 1  .  Goal: A
	    |   |---
	    |   | 
	    |   +
	    |   +
	 ?. |   | B   ..  Assumption  .  Setting up: v Elim: 1  .  Goal: A
	    |   |---
	    |   | 
	    |   | 
	    |   +
	    | A       .  v Elim: 1, 3-, ?-
	\endfitchproof
	\endanswer

\problem{6}
\argument
 A v B
 A > C
 B > D
\argumentline
 -D > C
\endargument
	\answer
	\fitchproof
	 1. | 
	\endfitchproof
	\endanswer

\problem{7}
\argument
 A v B
 C v -B
\argumentline
 A v C
\endargument
\Hint You'll have to do one $v$ Elim inside another.
	\answer
	\fitchproof
	 1. | 
	\endfitchproof
	\endanswer

\resetfitchsetup
\problem{8}
\argument
 [A ^ B] > [C > D]
 [-E ^ -F] > [B ^ -D]
 C ^ -E
\argumentline
 A > F
\endargument
\Hint Remember, when all else fails, use the last resort option.
	\answer
	\fitchproof
	 1. | 
	\endfitchproof
	\endanswer

\problem{9}
\argument
 -[-A ^ B]
 -[-B v C]
\argumentline
 A
\endargument
\Hint Problems 9 and 10 are much harder than anything on the exam.
	\answer
	\fitchproof
	 1. | 
	\endfitchproof
	\endanswer

\problem{10}
\argument
 A > B
\argumentline
 -A v B
\endargument
	\answer
	\fitchproof
	 1. | 
	\endfitchproof
	\endanswer

\endproblems
\bye
