\input logicdocument
\title{Phil 210 - Problem Set 3}

\heading
Problem Set 3
Your Name
Introduction to Formal Logic
\endheading

Construct truth tables for the following arguments. For each one, state whether it's valid or invalid.

\problems
\problem{Example}
\argument
 A ^ -B
 C v A
++++++++++
 -[C v B]
\endargument
        \answer
        \truthtable
         A | B | C : A. ^ .-.B | C. v .A | -.[.C. v .B.]
        +++++++++++++++++++++++++++++++++++++++++++++++++
         T | T | T : t.*F .F.t | t.*T .t |*F.[.t. T .t.]
         T | T | F : t.*F .F.t | f.*T .t |*F.[.f. T .t.]
         T | F | T : t.*T .T.f | t.*T .t |*F.[.t. T .f.]
         T | F | F : t.*T .T.f | f.*T .t |*T.[.f. F .f.]
         F | T | T : f.*F .F.t | t.*T .f |*F.[.t. T .t.]
         F | T | F : f.*F .F.t | f.*F .f |*F.[.f. T .t.]
         F | F | T : f.*F .T.f | t.*T .f |*F.[.t. T .f.]
         F | F | F : f.*F .T.f | f.*F .f |*T.[.f. F .f.]
        \endtruthtable

        Invalid. (You don't need to say why, but it's because row 3 is T T F.)
        \endanswer

\problem{1}
\argument
 -A
 -[B ^ A]
++++++++++
 -B
\endargument
        \answer
        \truthtable
         A | B : -.A | -.[.B. ^ .A.] | -.B
        +++++++++++++++++++++++++++++++++++
           |   :  .  |  .[. .   . .] |  . 
        \endtruthtable
        Valid/Invalid?
        \endanswer

\problem{2}
\argument
 A v B
 -A v C
 -B v C
++++++++
 C
\endargument
        \answer
        \truthtable
         A | B | C : A. v .B | -.A. v .C | -.B. v .C | C
        +++++++++++++++++++++++++++++++++++++++++++++++++
           |   |   :  .   .  |  . .   .  |  . .   .  |  
        \endtruthtable
        Valid/Invalid?
        \endanswer

\problem{3}
\argument
 -*[[A ^ B] v -A*]
 [-B v A] ^ B
+++++++++++++++++++
 A v -[B v A]
\endargument
        \answer
        \truthtable
         A | B : -.*[[.A. ^ .B.]. v .-.A.*] | [.-.B. v .A.]. ^ .B | A. v .-.[.B. v .A.]
        ++++++++++++++++++++++++++++++++++++++++++++++++++++++++++++++++++++++++++++++++
           |   :  . [[. .   . .].   . . . ] | [. . .   . .].   .  |  .   . .[. .   . .]
        \endtruthtable
        Valid/Invalid?
        \endanswer
\endproblems

For each of the following arguments,
\list
a: Construct a truth table to determine whether it's valid or invalid given only the connectives. State your decision.

b: If it's invalid, attempt to build a height model that's a counterexample. If the argument is valid, or if no countermodel can be constructed, just say so.

c: State whether the argument is valid or invalid, given both the connectives and the predicates.
\endlist

\problems
\problem{Example}
\argument
 Taller(ruth,alonzo) v Taller(ruth,kurt)
 -[Taller(alonzo,alonzo) ^ Taller(ruth,kurt)]
+++++++++++++++++++++++++++++++++++++++++++++
 Taller(ruth,kurt) v Taller(alonzo,kurt)
\endargument
        \answerlist
        a:
        \truthtable
         Tra | Trk | Tak : Tra. v .Trk | -.[.Tak. ^ .Trk.] | Trk. v .Tak
        +++++++++++++++++++++++++++++++++++++++++++++++++++++++++++++++++
          T  |  T  |  T  :  T .*T . T  |*F.[. t . T . t .] |  T .*T . T
          T  |  T  |  F  :  T .*T . T  |*T.[. f . F . t .] |  T .*T . F
          T  |  F  |  T  :  T .*T . F  |*T.[. t . F . f .] |  F .*T . T
          T  |  F  |  F  :  T .*T . F  |*T.[. f . F . f .] |  F .*F . F
          F  |  T  |  T  :  F .*T . T  |*F.[. t . T . t .] |  T .*T . T
          F  |  T  |  F  :  F .*T . T  |*T.[. f . F . t .] |  T .*T . F
          F  |  F  |  T  :  F .*F . F  |*T.[. t . F . f .] |  F .*T . T
          F  |  F  |  F  :  F .*F . F  |*T.[. f . F . f .] |  F .*F . F
        \endtruthtable
        Valid, given only the connectives.

         b:
         \heightmodel
          Alonzo < Ruth = Kurt
         \endheightmodel
         \OR
         \heightmodel
           Alonzo < Ruth < Kurt
         \endheightmodel

:        (You only need to give one countermodel, but I've given two for illustration.)

         c: Invalid, given the connectives and predicates.

:        (You don't need to say why, but it's because the height model depicts a situation in which $Tra$ is T, $Trk$ is F, and $Tak$ is F, and we know from the truth table that this makes both premises T and the conclusion F.)
                                                                                   \endanswerlist

\problem{4}
\argument
 Taller(ruth,alonzo) v -Taller(ruth,kurt)
 -[Taller(alonzo,kurt) ^ Taller(ruth,alonzo)]
++++++++++++++++++++++++++++++++++++++++++++++
 -Taller(alonzo,kurt)
\endargument
        \answerlist
        a:
        \truthtable
         Tra | Trk | Tak : Tra. v .-.Trk | -.[.Tak. ^ .Tra.] | -.Tak
        +++++++++++++++++++++++++++++++++++++++++++++++++++++++++++++
             |     |     :    .   . .    |  .[.   .   .   .] |  .  
        \endtruthtable
        Valid/Invalid?, given only the connectives.

        b: It's valid.\OR No countermodel is possible.\OR
        \heightmodel
         
        \endheightmodel

        c: Valid/Invalid?, given the connectives and predicates.
        \endanswerlist

\problem{5}
\argument
 Taller(ruth,alonzo) ^ Taller(alonzo,kurt)
 -Taller(ruth,alonzo) v Taller(alonzo,kurt)
++++++++++++++++++++++++++++++++++++++++++++
 Taller(ruth,kurt) v -Taller(alonzo,kurt) 
\endargument
        \answerlist
        a:
        \truthtable
         Tra | Tak | Trk : Tra. ^ .Tak | -.Tra. v .Tak | Trk. v .-.Tak
        +++++++++++++++++++++++++++++++++++++++++++++++++++++++++++++++
             |     |     :    .   .    |  .   .   .    |    .   . .  
        \endtruthtable
        Valid/Invalid?, given the connectives.

        b: It's valid.\OR No countermodel is possible.\OR
        \heightmodel
         
        \endheightmodel

        c: Valid/Invalid?, given the connectives and predicates.
        \endanswerlist

\endproblems
\bye
