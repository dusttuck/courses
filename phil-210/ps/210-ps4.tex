\input logicdocument
\title{Phil 210 - Problem Set 4}

\heading
Problem Set 4
Your Name
Introduction to Formal Logic
\endheading

Translate the following English sentences into the language of first-order logic.

\problems
\problem{1}
If Ruth is a logician, then she's happy.
        \answer
        $ $
        \endanswer

\problem{2}
Ruth is a logician if she's happy.
        \answer
        $ $
        \endanswer

\problem{3}
Ruth is a logician only if she's happy.
        \answer
        $ $
        \endanswer

\problem{4}
If Ruth and Alonzo are happy, then so is Kurt.
        \answer
        $ $
        \endanswer

\problem{5}
Ruth is happy only if either Alonzo or Kurt is.
        \answer
        $ $
        \endanswer

\problem{6}
If Ruth is a logician, then she'll only be happy if she's working.
        \answer
        $ $
        \endanswer

\problem{7}
Ruth is happy if and only if she's a logician.
        \answer
        $ $
        \endanswer

\problem{8}
If Ruth is happy, then she's not working.
        \answer
        $ $
        \endanswer

\problem{9}
Ruth is working unless she's happy.
        \answer
        $ $
        \endanswer

\problem{10}
Ruth is working unless she's a happy logician.
        \answer
        $ $
        \endanswer

\problem{11}
If either Ruth or Alonzo is working, then neither is happy.
        \answer
        $ $
        \endanswer

\problem{12}
Ruth isn't happy unless she's at work, and in that case, Alonzo is also working.
        \answer
        $ $
        \endanswer

\endproblems

Construct full (not abbreviated) truth tables for the following arguments. For each one, state whether it's valid or invalid.

\problems
\problem{13}
\argument
 A > -B
 -[C ^ B]
\argumentline
 C > -A
\endargument
        \answer
        \truthtable
         A | B | C : A. > .-.B | -.[.C. ^ .B.] | C. > .-.A
        \truthtableline
           |   |   :  .   . .  |  . . .   . .  |  .   . . 
        \endtruthtable
        Valid/Invalid?
        \endanswer

\problem{14}
\argument
 -*[[A > B] ^ -A*]
 [-B v A] > B
\argumentline
 A > -[B v A]
\endargument
        \answer
        \truthtable
         A | B : -.*[[.A. > .B.]. ^ .-.A.*] | [.-.B. v .A.]. > .B | A. > .-.[.B. v .A.]
        \truthtableline
           |   :  .   . .   . . .   . . .   |  . . .   . . .   .  |  .   . . . .   . 
        \endtruthtable
        Valid/Invalid?
        \endanswer

\endproblems

For the following argument,
\list
a: Use an abbreviated truth table to determine whether it's valid or invalid, given only the connectives. State your decision.

b: If it's invalid, attempt to build a height model that's a counterexample. If the argument is valid, or if no countermodel can be constructed, just say so.

c: State whether the argument is valid or invalid, given both the connectives and the predicates.
\endlist

\problems
\problem{15}
\argument
 -Shorter(ruth,alonzo)
 Shorter(ruth,kurt) > Shorter(ruth,alonzo)
\argumentline
 -[Shorter(alonzo,kurt) v Shorter(ruth,kurt)]
\endargument
        \answerlist
        a:
        \truthtable
         -.S(r,a) | S(r,k). > .S(r,a) | -.[.S(a,k). v .S(r,k).]
        \truthtableline
          .       |       .   .       |  . .      .   .
        \endtruthtable
        Valid/Invalid?, given only the connectives.

        b: It's valid.\OR No countermodel is possible.\OR
        \heightmodel
         
        \endheightmodel

        c: Valid/Invalid?, given the connectives and predicates.
        \endanswerlist

\endproblems

For the following argument, use an abbreviated truth table to determine whether it's valid or invalid. State your decision.

\problems
\problem{16}
\argument
 A v [B > C]
 C > D
 [B > D] > E
\argumentline
 -A > E
\endargument
        \answer
        \truthtable
         A. v .[.B. > .C.] | C. > .D | [.B. > .D.]. > .E | -.A. > .E
        \truthtableline
          .   . . .   . .  |  .   .  |  . .   . . .   .  |  . .   .
        \endtruthtable
        Valid/Invalid?
        \endanswer

\endproblems
\bye
