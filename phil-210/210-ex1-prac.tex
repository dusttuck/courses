\input logicdocument
\title{Phil 210 - Practice Exam 1}

\heading
Practice Exam 1
Your Name
Introduction to Formal Logic
\endheading

(50 points.) Translate the following English sentences into the language of first-order logic.

\problems
\problem{1} (16 points.)
Ruth is happy but not at work.
	\answer
	$ $
	\endanswer

\problem{2} (13 points.)
If Alonzo is a logician then so is Ruth.
	\answer
	$ $
	\endanswer

\problem{3} (10 points.)
Ruth is happy only if Alonzo is at work.
	\answer
	$ $
	\endanswer

\problem{4} (7 points.)
Neither Ruth nor Alonzo is a happy dog.
	\answer
	$ $
	\endanswer

\problem{5} (4 points.)
Alonzo is at work unless he's hungry, and in that case he's smiling.
	\answer
	$ $
	\endanswer

\endproblems

(20 points.) Construct truth tables for the following sentences and state which ones are equivalent.

\problems
\problem{6}
\list
a: $ A ^ -B $
b: $ B v [A ^ B] $
c: $ -[A > B] $
d: $ -[A ^ B] > B $
\endlist
	\answer
	\truthtable
	 A | B : A. ^ .-.B | B. v .[.A. ^ .B.] | -.[.A. > .B.] | -.[.A. ^ .B.]. > .B
	\truthtableline
	   |   :  .*  . .  |  .*  .[. .   . .] |* .[. .   . .] |  .[. .   . .].*  . 
	   |   :  .*  . .  |  .*  .[. .   . .] |* .[. .   . .] |  .[. .   . .].*  . 
	   |   :  .*  . .  |  .*  .[. .   . .] |* .[. .   . .] |  .[. .   . .].*  . 
	   |   :  .*  . .  |  .*  .[. .   . .] |* .[. .   . .] |  .[. .   . .].*  . 
	\endtruthtable
	Equivalences?
	\endanswer

\endproblems

(20 points.) Use a full or abbreviated truth table (your choice) to determine whether the following argument is valid or invalid and state your answer.

\problems
\problem{7}
\argument
 -[A > B]
 [C ^ B] > -[-A v C]
\argumentline
 C ^ -A
\endargument
	\answer
	\truthtable
	 A | B | C : -.[.A. > .B.] | [.C. ^ .B.]. > .-.[.-.A. v .C.] | C. ^ .-.B
	\truthtableline
	   |   |   :  .[. .   . .] | [. .   . .].   . .[. . .   . .] |  .   . . 
	\endtruthtable
	Valid/Invalid?
	\endanswer

\endproblems

(10 points.) Attempt to build a countermodel for the following argument to determine whether it is valid or invalid, given both the connectives predicates, then state your answer (and give the countermodel if possibe). You may again use whichever sort of truth table you want.

\problems
\problem{8}
\argument
 Shorter(alonzo,ruth) > Shorter(alonzo,kurt)
 -Shorter(alonzo,ruth) ^ Shorter(kurt,ruth)
\argumentline
 -Shorter(alonzo,kurt)
\endargument
	\answer
	\truthtable
	 S(a,r) | S(a,k) | S(k,r) : S(a,r). > .S(a,k) | -.S(a,r). ^ .S(k,r) | -.S(a,k)
	\truthtableline
	        |        |        :       .   .       |  .      .   .       |  .   
	\endtruthtable
	No countermodel is possible, so it's valid, given the connectives and predicates.
	\OR
	\heightmodel
	 
	\endheightmodel
	\endanswer

\endproblems
\bye
