\input logicdocument
\title{Phil 210 - Problem Set 7}

\heading
Problem Set 7
Your Name
Introduction to Formal Logic
\endheading

For each of the following valid arguments, complete a Fitch-style natural deduction to prove its validity. Do this by filling in the missing lines, citations, or both.

\problems
\problem{1}
\argument
 A ^ B
 B > C
\argumentline
 C v D
\endargument
	\answer
	\fitchproof
	 1. | A ^ B
	 2. | B > C  ..  Goal: C v D
	    |---
	 3. | B      .  ^ Elim: 1
	 4. |        .  > Elim: 2, 3
	 5. |        .  v Intro: 4
	\endfitchproof
	\endanswer

\problem{2}
\argument
 A ^ [B > D]
 [A v C] > B
\argumentline
 D
\endargument
	\answer
	\fitchproof
	 1. | A ^ [B > D]
	 2. | [A v C] > B  ..  Goal: D
	    |---
	 3. | 
	\endfitchproof
	\endanswer

\problem{3}
\argument
 A > B
 B > C
\argumentline
 A > [C v D]
\endargument
	\answer
	\fitchproof
	 1. | A > B
	 2. | B > C        ..  Goal: A > [C v D]
	    |---
	 3. |   | A        ..  Assumption  .  Setting up: > Intro  .  Goal: C v D
	    |   |---
	 4. |   | 
	\endfitchproof
	\endanswer

\problem{4}
\argument
 A > [B ^ C]
 [B v D] > E
\argumentline
 A > E
\endargument
	\answer
	\fitchproof
	 1. | 
	\endfitchproof
	\endanswer

\widerfitchsetup % This makes "Setting up: v Elim: #" fit better.
\problem{5}
\argument
 A > C
 B > C
 [D ^ A] v [E ^ B]
\argumentline
 C
\endargument
	\answer
	\fitchproof
	 1. | A > C
	 2. | B > C
	 3. | [D ^ A] v [E ^ B]  ..  Goal: C
	    |---
	 4. |   | D ^ A  ..  Assumption  .  Setting up: v Elim: 3  .  Goal: C
	    |   |---
	    |   | 
	    |   +
	    |   +
	 ?. |   | E ^ B  ..  Assumption  .  Setting up: v Elim: 3  .  Goal: C
	    |   |---
	    |   | 
	    |   +
	    | C          .  v Elim: 3, 4-, ?-
	\endfitchproof
	\endanswer

\problem{6}
\argument
 A > C
 B > D
\argumentline
 [A v B] > [C v D]
\endargument
	\answer
	\fitchproof
	 1. | A > C
	 2. | B > D          ..  Goal: [A v B] > [C v D]
	    |---
	 3. |   | A v B      ..  Setting up: > Intro    .  Goal: C v D
	    |   |---
	 4. |   | 
	\endfitchproof
	\endanswer

\problem{7}
\argument
 F(a) v G(a)
 F(b) > G(b)
\argumentline
 b=a > G(b)
\endargument
\Hint You can choose to set up $>$ Intro or $v$ Elim first. Both work.
	\answer
	\fitchproof
	 1. | 
	\endfitchproof
	\endanswer

\problem{8}
\argument
 A v B
 A > B
\argumentline
 B
\endargument
\Hint This is the first proof where you need to use Reit.
	\answer
	\fitchproof
	 1. | 
	\endfitchproof
	\endanswer

\problem{9}
\argument

\argumentline
 A > [B > A]
\endargument
	\answer\resetfitchsetup
	\fitchproof
	 1. | 
	\endfitchproof
	\endanswer

\problem{10}
\argument
 A v B
 A > C
 B > [F ^ G]
 C > [D > E]
\argumentline
 D > [E v G]
\endargument
\Hint Again, it doesn't matter if you set up $>$ Intro or $v$ Elim first.
	\answer
	\fitchproof
	 1. | 
	\endfitchproof
	\endanswer

\problem{11}
\argument
 A v B
 A > C
 [C > B] > D
 [C v D] > E
\argumentline
 E
\endargument
\Hint This requires something similar to problem 6 from last week.
	\answer
	\fitchproof
	 1. | 
	\endfitchproof
	\endanswer

\endproblems
\bye
