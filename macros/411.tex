\input eplain % This is included for verbatim, hyperlinks, and packages.
\input amssym.def % rather than amssym.tex, because we only need one symbol:
\newsymbol\square 1003
\beginpackages
\usepackage{color}
\endpackages


% ---------------
% General utility
% ---------------

%  Make @text@ be \verbatim text|endverbatim.
\def\at{@}
\catcode`\@=\active
\def@{\bgroup\red%
	\ifx\resetasterisk\undefined \else\resetasterisk \fi%
	\catcode`\^=\active%
	\def@{\endverbatim\egroup}%
	\verbatim\makecaretbetter%
}


%  Make the verbatim caret better.
\input rotate
\begingroup
\catcode`\^=\active
\gdef\makecaretbetter{%
	\catcode`\^=\active%
	\def^{{\setbox0=\hbox{\tt>}\rotl0\box0}}%
}
\endgroup


%  URLs
\enablehyperlinks[dvipdfm]
\hlopts{bwidth=0}
\def\url#1{\hlstart{url}{}{#1}\tt#1\hlend}


\def\ensuremath#1{\relax\ifmmode#1\else$#1$\fi}%  Ensure math mode
\def\ensureline{\hskip0pt}%  Make sure that the line has begun
\def\mystrut#1#2{\vrule height#1 depth#2 width0pt}%  Custom struts
\def\makereturnsspaces{\catcode`\^^M=10\relax}


% -------------
% Abbreviations
% -------------

\catcode`\>=12%      make > other
\mathcode`\>="3221%  \rightarrow
\catcode`\|=12
\mathcode`\|="3260%  \vdash
\catcode`\!=12
\mathcode`\!="023F%  \bot
\catcode`\-=12%
\mathcode`\-="023A%  \lnot
\catcode`\+=12%
\mathcode`\+="0\the\msafam03%  amssym.def \square; only works if msafam is 0-9, otherwise need to translate to hex first
\catcode`\^=12%
\mathcode`\^="0234%  \triangle
%\catcode`\?=12%
%\mathcode`\?="020F%  \bullet, but \mathord
\mathcode`\U="7255%  change family to 2
\mathcode`\P="7005%  \Pi
\mathcode`\L="0115%  \lambda
\mathcode`\G="7000%  \Gamma
\mathcode`\Y="7A59%  change family to A (10 - cmss)

\def\To{\qquad\ensuremath{=\!=\!\Rightarrow}\qquad}
\def\cv{\ensuremath{{}>_\beta{}}}
\def\rd{\ensuremath{\mathrel{\rlap{$>$}\,\mathord>_\beta{}}}}
\def\eq{\ensuremath{{}=_\beta{}}}
\def\cvs{\quad\cv\quad}


% ------
% Colors
% ------

\definecolor{red}{RGB}{255,77,92}
\definecolor{lightred}{RGB}{255,165,173}
\definecolor{darkred}{RGB}{200,49,62}
\definecolor{green}{RGB}{146,208,80}
\definecolor{lightgreen}{RGB}{200,231,167}
\definecolor{darkgreen}{RGB}{114,180,52}
\definecolor{blue}{RGB}{0,176,240}
\definecolor{lightblue}{RGB}{127,215,247}
\definecolor{darkblue}{RGB}{0,114,188}
\definecolor{gray}{RGB}{210,210,210}
\definecolor{lightgray}{RGB}{245,245,245}
\definecolor{darkgray}{RGB}{160,160,160}


% -----------
% Math colors
% -----------

\def\makecolorcs#1#2{\def#1{\let\curcolor#1\color{#2}}}

\makecolorcs\red{red}
\makecolorcs\green{green}
\makecolorcs\blue{blue}
\let\mlue\blue
\makecolorcs\black{black}
\let\mlack\black
\makecolorcs\gray{gray}
\makecolorcs\darkgray{darkgray}

%  This is a lazy hack and only works for a single prime following the variable
\def\makecolor#1#2#3{\bgroup#1#2%
	\if#3'%
		#3\egroup%
	\else%
		\egroup\expandafter#3%  \expandafter so that the second * in *v*v doesn't get eaten
	\fi}

\let\starcolor\green
\let\mathcolor\blue
\let\typecolor\green
\let\slashcolor\darkgray

\def\zz#1#2#3{\black#1#2#3\mathcolor}
\let\z\black

\def\blackmath{%
\let\starcolor\black%
\let\mathcolor\black%
\let\typecolor\black%
\let\slashcolor\black%
}

\def\sscript#1{\let\savecolor\typecolor\let\typecolor\curcolor_{#1}\let\typecolor\savecolor}
\catcode`\_=\active
\let_=\sscript

\begingroup
\catcode`\b=\active
\catcode`\[=\active
\catcode`\]=\active
\catcode`\/=\active
\catcode`\|=\active
\catcode`\:=\active
\catcode`\,=\active
%\catcode`\*=\active
\gdef\mathcolors{%
	\mathcolor%
	\catcode`\b=\active%
	\def b{{\mlack\mathchar"010C}}% \beta
	\catcode`\[=\active%
	\catcode`\]=\active%
	\catcode`\/=\active%
	\catcode`\|=\active%
	\catcode`\:=\active%
	\catcode`\,=\active%
	%\catcode`\*=\active%
	\def[{{\slashcolor\mathchar"405B}}%
	\def]{{\slashcolor\mathchar"505D}}%
	\def/{{\slashcolor\mathchar"013D}}%
	\def|{\slashcolor\mathchar"3260\mathcolor}% \vdash
	\def:{\typecolor\mathchar"303A }% colon
	\def,{\slashcolor\mathchar"313B\mathcolor}% comma as relation
	%\def*{\makecolor{\starcolor}}%
}
\endgroup

\everymath{\mathcolors\relax}


% ------
% Breaks
% ------

\def\pagebreak{\vfill\eject}
\def\linebreak{\hfill\break}


% -----
% Skips
% -----

\def\negskip{\vskip-10pt}
\def\medskip{\vskip5pt}
\def\bigskip{\vskip10pt}
\def\Bigskip{\vskip15pt}
\def\biggskip{\vskip20pt}
\def\Biggskip{\vskip30pt}


% ----------------------
% Mimicry of article.cls
% ----------------------

%\font\headingtitlefonts=cmr10 scaled \magstep3
\def\headingfonts{%
	\font\rm=cmr10 scaled \magstephalf%
	\font\tt=cmtt10 scaled \magstephalf%
	\rm%
}
\def\headingtitlefonts{%
	\font\rm=cmr10 scaled \magstep3%
	\font\tt=cmtt10 scaled \magstep3%
	\font\cmsy=cmsy10 scaled \magstep3%
	\textfont0=\rm%
	\textfont2=\cmsy%
	\rm%
}

\newif\ifheadingfirstrow
\def\headingrow{\ifheadingfirstrow\headingtitlefonts\mystrut{0pt}{18.5pt}\fi}
% 18.5pt = Bigskip + 3.5pt

\def\headinghalign{
	\headingfirstrowtrue
	\headingfonts
	\def\par{\crcr}
	\halign\bgroup
		\obeylines%
		\headingrow%
		\hfil##\hfil%
		\global\headingfirstrowfalse%
		\cr%
}

\newdimen\headingheight
\setbox0\vbox{
	\headinghalign
		\strut
		\strut
		\strut
	\egroup
	\biggskip
}
\headingheight=\ht0

\def\heading{\ensureline\hfil\vbox to\headingheight\bgroup\headinghalign}
\def\endheading{\par\egroup\vss\egroup}


% ------------
% Problem sets
% ------------

\let\problemfont=\bf
\let\problemspace=\quad
\def\problemprint#1{{\problemfont#1.\problemspace}}

\newdimen\problemindent
\def\setproblemindent#1{%
	\setbox0\hbox{\problemprint{#1}}%
	\problemindent=\wd0%
}
\setproblemindent{00}

\def\problems{\nobreak\par\bgroup%
	%\let\mathcolor\black%
	%\let\slashcolor\blue%
	\advance\leftskip by\problemindent%
	\def\Hint{\qquad{\bf Hint: }}%
	\def\points##1{\hskip1.5em(##1\ points)}%
}
\def\endproblems{\egroup\Biggskip}

\def\problem#1{\biggskip\noindent\llap{\problemprint{#1}}\ignorespaces\obeylines\gobble}


% -------
% Answers
% -------

\font\tensc=cmcsc10

\newdimen\answerindent \answerindent=2em

\def\answercore{\par\nobreak\bigskip\bgroup
	\parindent=0pt
	\def\OR{\hskip1\answerindent{\tensc or}\hskip1\answerindent}
	\advance\leftskip by\answerindent
	\obeylines
}
\def\answer{
	\answercore
}
\def\endanswer{\par\egroup}
\def\alternativeanswer{\endanswer\answer}


% -------------
% File settings
% -------------

\magnification\magstephalf
\raggedbottom
\overfullrule=0pt
\parindent=0pt
\interlinepenalty=10000


% These settings are specifically for Overleaf:
\special{papersize=8.5in,11in}
\def\title#1{}
\def\fitchproofneg{\fitchproofindentby{-4\fitchspace}}
